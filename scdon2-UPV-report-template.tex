\documentclass[mstat,12pt]{unswthesis}

\usepackage{color}
\usepackage{fancyvrb}
\newcommand{\VerbBar}{|}
\newcommand{\VERB}{\Verb[commandchars=\\\{\}]}
\DefineVerbatimEnvironment{Highlighting}{Verbatim}{commandchars=\\\{\}}
% Add ',fontsize=\small' for more characters per line
\usepackage{framed}
\definecolor{shadecolor}{RGB}{248,248,248}
\newenvironment{Shaded}{\begin{snugshade}}{\end{snugshade}}
\newcommand{\AlertTok}[1]{\textcolor[rgb]{0.94,0.16,0.16}{#1}}
\newcommand{\AnnotationTok}[1]{\textcolor[rgb]{0.56,0.35,0.01}{\textbf{\textit{#1}}}}
\newcommand{\AttributeTok}[1]{\textcolor[rgb]{0.77,0.63,0.00}{#1}}
\newcommand{\BaseNTok}[1]{\textcolor[rgb]{0.00,0.00,0.81}{#1}}
\newcommand{\BuiltInTok}[1]{#1}
\newcommand{\CharTok}[1]{\textcolor[rgb]{0.31,0.60,0.02}{#1}}
\newcommand{\CommentTok}[1]{\textcolor[rgb]{0.56,0.35,0.01}{\textit{#1}}}
\newcommand{\CommentVarTok}[1]{\textcolor[rgb]{0.56,0.35,0.01}{\textbf{\textit{#1}}}}
\newcommand{\ConstantTok}[1]{\textcolor[rgb]{0.00,0.00,0.00}{#1}}
\newcommand{\ControlFlowTok}[1]{\textcolor[rgb]{0.13,0.29,0.53}{\textbf{#1}}}
\newcommand{\DataTypeTok}[1]{\textcolor[rgb]{0.13,0.29,0.53}{#1}}
\newcommand{\DecValTok}[1]{\textcolor[rgb]{0.00,0.00,0.81}{#1}}
\newcommand{\DocumentationTok}[1]{\textcolor[rgb]{0.56,0.35,0.01}{\textbf{\textit{#1}}}}
\newcommand{\ErrorTok}[1]{\textcolor[rgb]{0.64,0.00,0.00}{\textbf{#1}}}
\newcommand{\ExtensionTok}[1]{#1}
\newcommand{\FloatTok}[1]{\textcolor[rgb]{0.00,0.00,0.81}{#1}}
\newcommand{\FunctionTok}[1]{\textcolor[rgb]{0.00,0.00,0.00}{#1}}
\newcommand{\ImportTok}[1]{#1}
\newcommand{\InformationTok}[1]{\textcolor[rgb]{0.56,0.35,0.01}{\textbf{\textit{#1}}}}
\newcommand{\KeywordTok}[1]{\textcolor[rgb]{0.13,0.29,0.53}{\textbf{#1}}}
\newcommand{\NormalTok}[1]{#1}
\newcommand{\OperatorTok}[1]{\textcolor[rgb]{0.81,0.36,0.00}{\textbf{#1}}}
\newcommand{\OtherTok}[1]{\textcolor[rgb]{0.56,0.35,0.01}{#1}}
\newcommand{\PreprocessorTok}[1]{\textcolor[rgb]{0.56,0.35,0.01}{\textit{#1}}}
\newcommand{\RegionMarkerTok}[1]{#1}
\newcommand{\SpecialCharTok}[1]{\textcolor[rgb]{0.00,0.00,0.00}{#1}}
\newcommand{\SpecialStringTok}[1]{\textcolor[rgb]{0.31,0.60,0.02}{#1}}
\newcommand{\StringTok}[1]{\textcolor[rgb]{0.31,0.60,0.02}{#1}}
\newcommand{\VariableTok}[1]{\textcolor[rgb]{0.00,0.00,0.00}{#1}}
\newcommand{\VerbatimStringTok}[1]{\textcolor[rgb]{0.31,0.60,0.02}{#1}}
\newcommand{\WarningTok}[1]{\textcolor[rgb]{0.56,0.35,0.01}{\textbf{\textit{#1}}}}


%%%%%%%%%%%%%%%%%%%%%%%%%%%%%%%%%%%%%%%%%%%%%%%%%%%%%%%%%%%%%%%%%%
% 
% OK...Now we get to some actual input.  The first part sets up
% the title etc that will appear on the front page
%
%%%%%%%%%%%%%%%%%%%%%%%%%%%%%%%%%%%%%%%%%%%%%%%%%%%%%%%%%%%%%%%%%

\title{Projet réalisé par l'équipe 2\\[0.5cm]Rapport de groupe en
Sciences des Données 2 + Bases de données}

\authornameonly{Girondin Audric 22001931 Duckes Jonathan 22001974 Mendil
Youcef 201810962 Mohand-Amer Manel 201810962 }

\author{\Authornameonly}

\copyrightfalse
\figurespagefalse
\tablespagefalse

%%%%%%%%%%%%%%%%%%%%%%%%%%%%%%%%%%%%%%%%%%%%%%%%%%%%%%%%%%%%%%%%%
%
%  And now the document begins
%  The \beforepreface and \afterpreface commands puts the
%  contents page etc in
%
%%%%%%%%%%%%%%%%%%%%%%%%%%%%%%%%%%%%%%%%%%%%%%%%%%%%%%%%%%%%%%%%%%


\input{header.tex}

\renewcommand{\contentsname}{Table des matières}

\renewcommand{\chaptername}{Chapitre}



\begin{document}

\beforepreface

%\afterpage{\blankpage}

% plagiarism

\prefacesection{Déclaration de non plagiat}

\vskip 2pc \noindent Nous déclarons que ce rapport est le fruit de notre seul travail, à part lorsque cela est indiqué  explicitement. 

\vskip 2pc  \noindent Nous acceptons que la personne évaluant ce rapport puisse, pour les besoins de cette évaluation:
\begin{itemize}
\item la reproduire et en fournir une copie à un autre membre de l'université; et/ou,
\item en communiquer une copie à un service en ligne de détection de plagiat (qui pourra en retenir une copie pour les besoins d'évaluation future).
\end{itemize}

\vskip 2pc \noindent Nous certifions que nous avons lu et compris les règles ci-dessus.\vspace{24pt}

\vskip 2pc \noindent En signant cette déclaration, nous acceptons ce qui précède.
\vskip 2pc \noindent
Signature: \rule{7cm}{0.25pt} \hfill Date: \rule{4cm}{0.25pt} \\[1cm]
Signature: \rule{7cm}{0.25pt} \hfill Date: \rule{4cm}{0.25pt} \\[1cm]
Signature: \rule{7cm}{0.25pt} \hfill Date: \rule{4cm}{0.25pt} \\[1cm]
Signature: \rule{7cm}{0.25pt} \hfill Date: \rule{4cm}{0.25pt} \\[1cm]
\vskip 1pc

%\afterpage{\blankpage}

% Acknowledgements are optional


\prefacesection{Remerciements}

{\bigskip}Nos plus sincères remerciements vont à nos encadrants
pédagogiques pour les conseils avisés sur notre travail.\\[1cm] 

{\bigskip\bigskip\bigskip\noindent} 12/12/2022.

%\afterpage{\blankpage}

% Abstract

\prefacesection{Résumé}



%\afterpage{\blankpage}


\afterpreface





%%%%%%%%%%%%%%%%%%%%%%%%%%%%%%%%%%%%%%%%%%%%%%%%%%%%%%%%%%%%%%%%%%
%
% Now we can start on the first chapter
% Within chapters we have sections, subsections and so forth
%
%%%%%%%%%%%%%%%%%%%%%%%%%%%%%%%%%%%%%%%%%%%%%%%%%%%%%%%%%%%%%%%%%%



%%%%%%%%%%%%%%%%%%%%%%%%%%%%%%%%%%%%%

%\afterpage{\blankpage}


\hypertarget{introduction}{%
\chapter{Introduction}\label{introduction}}

Suite à la récente pandémie survenue ces dernieres années à cause du
Covid-19, nous nous sommes intéressés au domaine de la santé, plus
précisement dans la région d'Outre-mer. Durant cette période de
nombreuses personnes sont tombées malades, tous les hopitaux ont été
mobilisés ainsi qu'énormément de personnels mais il y a aussi eu une
forte quantité de médicaments vendus et du coup remboursés, c'est
pourquoi notre objet d'études portera sur :

\bigskip

\centering

\textbf{Quels sont les médicaments les plus ou les mieux remboursés en
région d'Outre-mer ?}

\bigskip

\justifying

\hypertarget{quelques-duxe9tails-techniques}{%
\section{Quelques détails
techniques}\label{quelques-duxe9tails-techniques}}

la Figure\(~\)\ref{myfigure}.

\begin{figure}
\hypertarget{myfigure}{%
\centering
\includegraphics[width=4cm,height=2cm]{logo-upvm_4.jpg}
\caption{Une légende sous la figure.}\label{myfigure}
}
\end{figure}

\hypertarget{base-de-donnuxe9es}{%
\chapter{Base de données}\label{base-de-donnuxe9es}}

\hypertarget{descriptif-des-tables}{%
\section{Descriptif des tables}\label{descriptif-des-tables}}

Tables sélectionnées :

Médicament :

ATC1 : Groupe Principal Anatomique CIP13 : Code Identification
Spécialité Pharmaceutique GEN\_NUM : Groupe Générique

Bénéficiaire :

BEN\_REG : Région de Résidence du Bénéficiaire (Filtrée de façon à
seulement prendre la région Outre-Mer)

Prescripteur :

PSP\_SPE : Prescripteur

Indicateur :

REM : Montant Remboursé BSE : Base de Remboursement BOITES : Nombre de
boîtes délivrées

\hypertarget{moduxe8les-mcd-et-mod}{%
\section{Modèles MCD et MOD}\label{moduxe8les-mcd-et-mod}}

Ci-dessous notre MCD et MOD : \bigskip

Le MCD,Figure\(~\)\ref{MCD}

\begin{figure}
\hypertarget{MCD}{%
\centering
\includegraphics[width=10cm,height=10cm]{C:/Users/audri/Desktop/Licence_MIASHS/L3_MIASHS_AUDRIC/Base de données/Projet/mcd/mcd.png}
\caption{LE MCD}\label{MCD}
}
\end{figure}

Le MOD,Figure\(~\)\ref{MOD}

\begin{figure}
\hypertarget{MOD}{%
\centering
\includegraphics{C:/Users/audri/Desktop/Licence_MIASHS/L3_MIASHS_AUDRIC/Base de données/Projet/MOD_modifie.png}
\caption{LE MOD}\label{MOD}
}
\end{figure}

\hypertarget{import-des-donnuxe9es}{%
\section{Import des données}\label{import-des-donnuxe9es}}

\begin{itemize}
\tightlist
\item
  Suppressions de valeurs manquantes
\item
  Suppression de colonnes inutiles pour notre problématique
\item
  Tri par région afin d'étudier uniquement la région Outre-mer
\end{itemize}

\hypertarget{requuxeates-ruxe9alisuxe9es}{%
\section{Requêtes réalisées}\label{requuxeates-ruxe9alisuxe9es}}

Voici les différentes requêtes réalisées au cours de notre projet, voir
les figures ci dessous:

\begin{itemize}
\tightlist
\item
  Le montant total vendu par medicament
\end{itemize}

\includegraphics{C:/Users/audri/Desktop/Licence_MIASHS/L3_MIASHS_AUDRIC/Base de données/Projet/ScreenshotsRequetes_nouvelle_base/ScreenshotsRequetes_nouvelle base/Afficher le montant total vendu par medicament/Requete- montant total vendu.png}
\bigskip
\includegraphics{C:/Users/audri/Desktop/Licence_MIASHS/L3_MIASHS_AUDRIC/Base de données/Projet/ScreenshotsRequetes_nouvelle_base/ScreenshotsRequetes_nouvelle base/Afficher le montant total vendu par medicament/Resultat- montant total vendu.png}

\bigskip

\begin{itemize}
\tightlist
\item
  Lister les médicaments non remboursés
\end{itemize}

\includegraphics{C:/Users/audri/Desktop/Licence_MIASHS/L3_MIASHS_AUDRIC/Base de données/Projet/ScreenshotsRequetes_nouvelle_base/ScreenshotsRequetes_nouvelle base/Medicaments non remboursés/Requete - les medicaments non remboursés.png}\hspace*{2cm}
\includegraphics{C:/Users/audri/Desktop/Licence_MIASHS/L3_MIASHS_AUDRIC/Base de données/Projet/ScreenshotsRequetes_nouvelle_base/ScreenshotsRequetes_nouvelle base/Medicaments non remboursés/Resultat - les medicaments non remboursés.png}

\begin{itemize}
\tightlist
\item
  Les médicaments les plus souvent remboursés
\end{itemize}

\includegraphics{C:/Users/audri/Desktop/Licence_MIASHS/L3_MIASHS_AUDRIC/Base de données/Projet/ScreenshotsRequetes_nouvelle_base/ScreenshotsRequetes_nouvelle base/Les plus souvent remboursés/Requete- les plus souvent remboursés.png}
\bigskip

\includegraphics{C:/Users/audri/Desktop/Licence_MIASHS/L3_MIASHS_AUDRIC/Base de données/Projet/ScreenshotsRequetes_nouvelle_base/ScreenshotsRequetes_nouvelle base/Les plus souvent remboursés/Resultat- les plus souvent remboursés.png}

\begin{itemize}
\tightlist
\item
  Les moins souvent remboursés
\end{itemize}

\includegraphics{C:/Users/audri/Desktop/Licence_MIASHS/L3_MIASHS_AUDRIC/Base de données/Projet/ScreenshotsRequetes_nouvelle_base/ScreenshotsRequetes_nouvelle base/les moins souvent remboursés/Requete- les moins souvent remboursés.png}
\bigskip

\includegraphics{C:/Users/audri/Desktop/Licence_MIASHS/L3_MIASHS_AUDRIC/Base de données/Projet/ScreenshotsRequetes_nouvelle_base/ScreenshotsRequetes_nouvelle base/les moins souvent remboursés/Resultat - les moins souvent remboursés.png}

\begin{itemize}
\tightlist
\item
  Lister le prix et le montant remboursé de chaque médicament
\end{itemize}

\includegraphics{C:/Users/audri/Desktop/Licence_MIASHS/L3_MIASHS_AUDRIC/Base de données/Projet/ScreenshotsRequetes_nouvelle_base/ScreenshotsRequetes_nouvelle base/Afficher pour chaque médicament son prix et son montant remboursé/Requete-prix et montant remboursé.png}
\bigskip

\includegraphics{C:/Users/audri/Desktop/Licence_MIASHS/L3_MIASHS_AUDRIC/Base de données/Projet/ScreenshotsRequetes_nouvelle_base/ScreenshotsRequetes_nouvelle base/Afficher pour chaque médicament son prix et son montant remboursé/Resultat-prix et montant remboursé.png}

\bigskip

\begin{itemize}
\tightlist
\item
  Le Taux de remboursement
\end{itemize}

\includegraphics{C:/Users/audri/Desktop/Licence_MIASHS/L3_MIASHS_AUDRIC/Base de données/Projet/ScreenshotsRequetes_nouvelle_base/ScreenshotsRequetes_nouvelle base/Taux de remboursement/Requete- taux de remboursement.png}
\bigskip

\includegraphics{C:/Users/audri/Desktop/Licence_MIASHS/L3_MIASHS_AUDRIC/Base de données/Projet/ScreenshotsRequetes_nouvelle_base/ScreenshotsRequetes_nouvelle base/Taux de remboursement/Resultat-taux de remboursement.png}

\begin{itemize}
\tightlist
\item
  Les 10 médicaments les mieux remboursés \bigskip
\end{itemize}

\includegraphics[width=15cm,height=10cm]{C:/Users/audri/Desktop/Licence_MIASHS/L3_MIASHS_AUDRIC/Base de données/Projet/Les_10_medicaments_les_mieux_rembourses/Les 10 medicaments les mieux remboursés/Requete-Les 10 medicaments les mieux remboursés.png}
\bigskip

\includegraphics[width=15cm,height=10cm]{C:/Users/audri/Desktop/Licence_MIASHS/L3_MIASHS_AUDRIC/Base de données/Projet/Les_10_medicaments_les_mieux_rembourses/Les 10 medicaments les mieux remboursés/Resultat-Les 10 medicaments les mieux remboursés.png}

\bigskip

\hypertarget{quelques-duxe9tails-techniques-1}{%
\section{Quelques détails
techniques}\label{quelques-duxe9tails-techniques-1}}

Il est possible d'établir une connection entre Rstudio et PhpMyAdmin en
local à l'aide du code suivant:

\scriptsize

\begin{Shaded}
\begin{Highlighting}[]
\CommentTok{\# install.packages("RMySQL")}
\CommentTok{\# install.packages("DBI")}
\FunctionTok{library}\NormalTok{(DBI)}
\NormalTok{con }\OtherTok{\textless{}{-}}\NormalTok{ DBI}\SpecialCharTok{::}\FunctionTok{dbConnect}\NormalTok{(RMySQL}\SpecialCharTok{::}\FunctionTok{MySQL}\NormalTok{(),}
\AttributeTok{host =} \StringTok{"127.0.0.1"}\NormalTok{,}
\AttributeTok{port =} \DecValTok{3306}\NormalTok{,}
\AttributeTok{username =} \StringTok{"root"}\NormalTok{,}
\AttributeTok{password =} \StringTok{""}\NormalTok{,}
\AttributeTok{dbname =} \StringTok{"projet"}\NormalTok{)}
\end{Highlighting}
\end{Shaded}

\bigskip
\bigskip
\bigskip

Nous avons utilisé cette méthode afin d'importer notre base de données
sur Rstudio et créer des requetes ainsi que des graphiques.

\hypertarget{matuxe9riel-et-muxe9thodes}{%
\chapter{Matériel et Méthodes}\label{matuxe9riel-et-muxe9thodes}}

\hypertarget{logiciels}{%
\section{Logiciels}\label{logiciels}}

Nous avons utilisés principalement le langage de programmation Rstudio,
Wamp mais aussi Excel.

\begin{itemize}
\tightlist
\item
  Rstudio pour les analyses statistiques et la création du rapport à
  travers RMarkdown
\item
  Wamp afin de se connecter à PhpMyAdmin afin de travailler sur nos
  différentes requêtes
\item
  Excel pour effectuer le pré-traitement des données
\item
  Whatsapp, une application de messagerie instantanée afin de
  communiquer sur les avancées
\end{itemize}

Nous avons travaillé sur 4 ordinateurs differents :

\begin{itemize}
\item
  Swift SF113-31, processeur Intel(R) Pentium(R) CPU N4200 1.10 GHz,
  Mémoire RAM installée :4,00,Go (3,84,Go utilisable) Type du système :
  Système d'exploitation 64bits,processeur x64, Windows 10
\item
  Dell XPS 13 7390 2-in-1,processeur Intel(R) Core(TM) i7-1065G7 CPU @
  1.30GHz 1.50 GHz, Mémoire RAM installée:16,0 Go (15,8 Go utilisable)
  Type du système : Système d'exploitation 64 bits, processeur x64,
  Windows 11
\item
  DESKTOP-SI1AM2U AMD A9-9425 RADEON R5, 5 COMPUTE CORES 2C + 3G,
  3.10GHz, Mémoire RAM : 8,00(7,47 utilisable) Type du système : Système
  d'exploitation 64 bits, processeur x64, Windows 10
\item
  MacBook Pro(13-inch,2017,Rwo Thunderbolt 3 ports), Processeur : 2,3
  GHz Intel Core i5 double coeur, Mémoire 8Go 2133 MHz LPDDR3, macOS
  Monterey(version 12.6.1)
\end{itemize}

\bigskip

\hypertarget{description-des-donnuxe9es}{%
\section{Description des Données}\label{description-des-donnuxe9es}}

Les données sont stockés sur PhpMyAdmin dans la base de données sur un
serveur en .sql et sur R elles sont importées sous forme de dataframe.
Le fichier comporte 3 tables d'environ 45 000 lignes.

\hypertarget{nettoyage-des-donnuxe9es}{%
\section{Nettoyage des données}\label{nettoyage-des-donnuxe9es}}

En ce qui concerne notre base de données nous n'avons décidé de
supprimer les valeurs manquantes cependant il y avait beaucoup de
colonnes inutiles à l'analyse de données, donc nous avons décidés de les
supprimer.

\hypertarget{uxe9tapes-de-pruxe9-traitements}{%
\section{Étapes de
Pré-traitements}\label{uxe9tapes-de-pruxe9-traitements}}

Quelles transformations avez-vous effectuées sur vos données pour les
rendre utilisables? Tout d'abord notre jeu de donnée était composé de
plus de d'1 million de lignes.En filtrant par les régions et en ne
gardant que la région Outre-Mer cela nous a permis de réduire la base à
peu près 45 000 lignes.

\medskip
\centering

A l'aide du logiciel R, nous avons gardé uniquement les lignes des clés
existantes dans toutes les tables,nous avons utilisé des jointures
internes.

\medskip
\centering

De plus nous avons concaténé deux colonnes afin de créer une clé
unique,ci- dessous :

\medskip

\includegraphics{C:/Users/audri/Desktop/Licence_MIASHS/L3_MIASHS_AUDRIC/Base de données/Projet/création CIP7.jpeg}

\hypertarget{moduxe9lisation-de-la-base-de-donnuxe9es}{%
\section{Modélisation de la base de
données}\label{moduxe9lisation-de-la-base-de-donnuxe9es}}

\begin{figure}
\hypertarget{MCD}{%
\centering
\includegraphics[width=10cm,height=10cm]{C:/Users/audri/Desktop/Licence_MIASHS/L3_MIASHS_AUDRIC/Base de données/Projet/mcd/mcd.png}
\caption{LE MCD}\label{MCD}
}
\end{figure}

\hypertarget{moduxe9lisation-statistique}{%
\section{Modélisation statistique}\label{moduxe9lisation-statistique}}

\bigskip

Nous allons utiliser des diagrammes en barre, ainsi que des boxplots
afin d'avoir une représentation cohérente avec nos types de variables
mais aussi afin d'avoir des informations sur les données que nous
étudions et qui sont propre à nos requêtes

\hypertarget{analyse-exploratoire-des-donnuxe9es}{%
\chapter{Analyse Exploratoire des
Données}\label{analyse-exploratoire-des-donnuxe9es}}

Ici , nous avons des statisques basqiues afin de nous informer sur
données de la requete, elles ont été faite à laide de la fonction
``summary''

\medskip

\includegraphics[width=10cm,height=5cm]{C:/Users/audri/Desktop/Licence_MIASHS/L3_MIASHS_AUDRIC/Base de données/Projet/d2}

\medskip

\includegraphics[width=10cm,height=10cm]{C:/Users/audri/Desktop/Licence_MIASHS/L3_MIASHS_AUDRIC/Base de données/Projet/barplot.JPG}

\bigskip
\bigskip
\bigskip

\bigskip
\bigskip
\bigskip

\hypertarget{utiliser-r}{%
\section{Utiliser R}\label{utiliser-r}}

\centering

Nous pouvons utliser R afin de montrer une partie de code utilisé par
exemple :

\begin{verbatim}
l_ATC1<-c("Anti-infectieux","Dermatologie","Hormones systémiques","Sang & organes ","Sys cardio-vasculaire","Système digestif","Sys génito-urinaire")

somme montant remb<- c(1122911.0,397021.4,1516079.4,3695387.4,7511313.5,9894111.6,3423083.5)

barplot(somme montant remb,main = 'Barplot du montant total remboursé par zone ciblés de médiacement',xlab= "Zone ciblée",ylab = "Somme des montants remboursés",xlim =c(0,8) ,ylim= c(0,10000000), names.arg = l_ATC1,col ="deeppink",border = "royalblue",width = 1,space =NULL)
\end{verbatim}

\bigskip
\bigskip
\bigskip
\centering

Le code ci-dessus a généré le graphique précédent \bigskip

\hypertarget{analyse-et-ruxe9sultats}{%
\chapter{Analyse et Résultats}\label{analyse-et-ruxe9sultats}}

\hypertarget{un-premier-moduxe8le}{%
\section{Un premier modèle}\label{un-premier-moduxe8le}}

\bigskip
\bigskip

Voici l'un de nos premier modèle afin de déterminer les quartiles, le
minimum, le maximum , écart inter-quartile ainsi que la médiane d'une
requête. \bigskip

\bigskip
\bigskip

\includegraphics{C:/Users/audri/Desktop/Licence_MIASHS/L3_MIASHS_AUDRIC/Base de données/Projet/boxplot(somme)}

\bigskip
\bigskip

\hypertarget{quelques-exemples-de-ruxe9sultats-attendus}{%
\section{Quelques exemples de résultats
attendus}\label{quelques-exemples-de-ruxe9sultats-attendus}}

\bigskip
\centering

Boxplot des remboursements par catégorie \bigskip

\bigskip

\includegraphics[width=10cm,height=10cm]{C:/Users/audri/Desktop/Licence_MIASHS/L3_MIASHS_AUDRIC/Base de données/Projet/boxplot(remb par catégorie).JPG}

\medskip

Ci-dessous un diragramme représentant le taux de remboursement des
médicaments les plus vendus en Outre-mer \bigskip

\bigskip
\bigskip

\includegraphics[width=10cm,height=10cm]{C:/Users/audri/Desktop/Licence_MIASHS/L3_MIASHS_AUDRIC/Base de données/Projet/diag_tx_remb.png}

\hypertarget{discussion}{%
\chapter{Discussion}\label{discussion}}

Nous pouvons donc constater que le groupe ciblé, système digestif et
métabolisme, est celui qui a le plus grand montant remboursé sur cette
année 2021.Cependant nous avons aussi remarqué qu'il y avait une forte
relation entre le taux de remboursement et le prix du médicament.
\bigskip \bigskip

\bigskip
\centering

Avec les différentes requetes réalisé, nous avons pu voir aussi
l'évolution des prix des médicaments, et les différences de prix selon
le type de médicament, le type de zone ciblée ainsi que les différences
de remboursement.

\hypertarget{conclusion-et-perspectives}{%
\chapter{Conclusion et perspectives}\label{conclusion-et-perspectives}}

Nous pouvons donc en conclure que à l'aide des plusieurs requêtes et
diagrammes que les médicaments les mieux remboursés en 2021 sont ceux
traitant le système digestif et le métabolisme. On pourrait aussi dire
que le taux de remboursement a une certaine relation avec le prix de
vente des médicaments

\bigskip
\bigskip
\bigskip
\centering

Nous aurions pu envisager d'étudier cette problématique sur plusieurs
années par exemple sur les 5 ou les 10 dernieres années. Mais dans
l'avenir nous pouvons penser à faire une projections sur les années
futures, des estimations basées sur les années actuelles.

\hypertarget{bibliographie}{%
\chapter*{Bibliographie}\label{bibliographie}}
\addcontentsline{toc}{chapter}{Bibliographie}

\hypertarget{refs}{}
\begin{CSLReferences}{0}{0}
\end{CSLReferences}

\bigskip
\bigskip
\centering

GroupReportTemplate \bigskip \bigskip \centering ``Le Logiciel R'' par
Pierre Lafaye de Micheaux,Remy Drouilhet et Benoit Liquet \bigskip
\centering Cours de Rstudio dispensé à l'Université Paul-Valery 3 de
Montpellier \bigskip \centering Cours et TD de Programmation Web par
Sandra Bringuay,dispensé à l'Université Paul-Valery 3 de Montpellier
\bigskip \centering R Markdown Cheat Sheet

\bibliographystyle{elsarticle-harv}
\bibliography{references}

\hypertarget{annexes}{%
\chapter*{Annexes}\label{annexes}}
\addcontentsline{toc}{chapter}{Annexes}

\hypertarget{codes}{%
\section*{\texorpdfstring{\textbf{Codes}}{Codes}}\label{codes}}
\addcontentsline{toc}{section}{\textbf{Codes}}

\begin{verbatim}
ggplot(df3, aes(x=NomMedoc)) +
  geom_bar(aes(y=coutApresRemboursement), fill='blue', stat="identity") +
  scale_y_continuous(limits = c(0,10))+
  geom_point(aes(y=(NbBoites/10000)), color = rgb(0, 1, 0), pch=16, size=3) +
  geom_path(aes(y=coutApresRemboursement, group=1), colour="slateblue1", lty=3, size=0.9) +
  theme(axis.text.x = element_text(angle=90, vjust=0.6,size=7)) +
  labs(title = "Répartition du taux de remboursement des médicament les plus vendus en Outre-mer",  x = 'Nom des médicaments', y ='Taux de remboursement')
\end{verbatim}

\bigskip
\bigskip

\begin{verbatim}
  mutate(class = fct_reorder(l_ATC1, REM, .fun='length' )) %>%
  ggplot( aes(x=l_ATC1, y=REM, fill=l_ATC1)) + 
  geom_boxplot() +
  scale_y_continuous(limits = c(0,250))+
  xlab("l_ATC1") +
  theme(legend.position="none",axis.text.x = element_text(angle=-20, vjust=0.6,size=10)) +
  xlab("") +
  xlab("")
  
\end{verbatim}

\hypertarget{tables}{%
\section*{\texorpdfstring{\textbf{Tables}}{Tables}}\label{tables}}
\addcontentsline{toc}{section}{\textbf{Tables}}

Nous n'avons aucun tableaux en supplément à afficher.







\end{document}

